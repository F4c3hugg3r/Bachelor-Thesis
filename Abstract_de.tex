% !TEX root = ../Thesis.tex
%%
%%  Hochschule für Technik und Wirtschaft Berlin --  Abschlussarbeit
%%
%%  Abstract - Deutsch
%%
%%%%%%%%%%%%%%%%%%%%%%%%%%%%%%%%%%%%%%%%%%%%%%%%%%%%


\section*{Kurzfassung}
%Motivation, Fragestellung, Methodik, Ergebnisse, Schlussfolgerungen

Das horizontale Scannen von Netzwerken oder Adressräumen ist eine fundamentale Methode der proaktiven Sicherheitsforschung. 
Etablierte Hochleistungsscanner wie ZMap oder Masscan basieren 
überwiegend auf C, was zwar maximale \eng{Performance} ermöglicht, jedoch aufgrund fehlender Speichersicherheit Risiken für 
Sicherheitslücken birgt. Diese Bachelorarbeit untersucht, inwieweit ein in Rust implementierter Scanner hinsichtlich 
Durchsatz und Ressourceneffizienz mit diesen Tools konkurrieren kann und dabei durch die sprach-eigenen Garantien ein 
intrinsisch höheres Sicherheitsniveau bietet.

Hierfür wurde ein asynchroner \texttt{SYN}-Scanner (\enquote{SYN-Rust}) entwickelt, der moderne Linux-Kernel-Schnittstellen wie \texttt{AF\_XDP} und 
\texttt{eBPF} nutzt, um den Netzwerkstack partiell zu umgehen. Ergänzend kommt im \eng{User-Space} eine logisch entkoppelte Architektur 
unter Verwendung der \texttt{tokio}-Laufzeitumgebung zum Einsatz, die eine effiziente Nebenläufigkeit gewährleistet.
In einer kontrollierten Gigabit-Testumgebung wurde der Prototyp gegen ZMap und Masscan evaluiert.

Die Ergebnisse zeigen, dass die Rust-Implementierung im \eng{Zero-Copy}-Modus die Bandbreitengrenze der 
Gigabit-Schnittstelle vollständig ausschöpft. Besonders hervorzuheben ist die Ressourceneffizienz: SYN-Rust 
verarbeitete pro CPU-Auslastungsprozent etwa dreimal mehr Pakete als Masscan und viermal mehr als ZMap. 
Die Arbeit belegt somit, dass Rust in Kombination mit modernen Kernel-Mechanismen eine leistungsfähige und sichere 
Alternative zu C für die Entwicklung systemnaher Netzwerkanwendungen im Bereich des Netzwerkscannings darstellt.
% !TEX root = ../Thesis.tex
%%
%%  Hochschule für Technik und Wirtschaft Berlin --  Abschlussarbeit
%%
%%  Abstract - Deutsch
%%
%%%%%%%%%%%%%%%%%%%%%%%%%%%%%%%%%%%%%%%%%%%%%%%%%%%%


\section*{Kurzfassung}
%Motivation, Fragestellung, Methodik, Ergebnisse, Schlussfolgerungen

Das horizontale Scannen von Netzwerken oder Adressräumen ist eine fundamentale Methode der proaktiven Sicherheitsforschung. 
Etablierte Hochleistungsscanner wie ZMap oder Masscan basieren 
überwiegend auf C. Dies ermöglicht zwar maximale Performance, birgt jedoch aufgrund manueller Speicherverwaltung Risiken für Sicherheitslücken. 

Diese Bachelorarbeit untersucht, inwieweit ein in Rust implementierter Scanner hinsichtlich 
Durchsatz und Ressourceneffizienz mit diesen Tools konkurrieren kann und dabei durch die sprach-eigenen Garantien ein 
intrinsisch höheres Sicherheitsniveau bietet.

Hierfür wurde ein asynchroner \texttt{SYN}-Scanner (\enquote{SYN-Rust}) entwickelt, der moderne Linux-Kernel-Schnittstellen wie \texttt{AF\_XDP} und 
\texttt{eBPF} nutzt, um den Netzwerkstack partiell zu umgehen. Die Softwarearchitektur im User-Space basiert auf einer strikten Entkopplung der 
Funktionskomponenten. Unterstützt durch die asynchrone \texttt{tokio}-Laufzeitumgebung ermöglicht dieses Design eine effiziente, nebenläufige Verarbeitung.
In einer kontrollierten Gigabit-Testumgebung wurde der Prototyp gegen ZMap und Masscan evaluiert.

Die Ergebnisse zeigen, dass die Rust-Implementierung im Zero-Copy-Modus die physikalische Bandbreite vollständig ausschöpft. 
Besonders hervorzuheben ist die Ressourceneffizienz: SYN-Rust 
verarbeitete pro CPU-Auslastungsprozent etwa dreimal mehr Pakete als Masscan und viermal mehr als ZMap. 
Die Arbeit belegt somit, dass Rust in Kombination mit modernen Kernel-Mechanismen eine leistungsfähige und sichere 
Alternative zu C für die Entwicklung systemnaher Netzwerkscanner darstellt.
% !TEX root = ../Thesis.tex
%%
%%  Hochschule für Technik und Wirtschaft Berlin --  Abschlussarbeit
%%
%%  Abstract - Englisch
%%
%%%%%%%%%%%%%%%%%%%%%%%%%%%%%%%%%%%%%%%%%%%%%%%%%%%%


\section*{Abstract}

Horizontal scanning of networks or address spaces constitutes a fundamental method in proactive security research. Established 
high-performance scanners such as ZMap or Masscan are predominantly based on C. While this enables maximum performance, it entails 
risks regarding security vulnerabilities due to manual memory management.

This bachelor thesis investigates to what extent a scanner implemented in Rust can compete with these tools in terms of throughput 
and resource efficiency, while offering an intrinsically higher level of security through the language's inherent guarantees. 
For this purpose, an asynchronous \texttt{SYN} scanner (\enquote{SYN-Rust}) was developed, which utilizes modern Linux kernel 
interfaces such as \texttt{AF\_XDP} and \texttt{eBPF} to partially bypass the network stack. The software architecture in user-space 
is based on a strict decoupling of functional components. Supported by the asynchronous \texttt{tokio} runtime, this design facilitates 
efficient, concurrent processing.

The prototype was evaluated against ZMap and Masscan in a controlled gigabit test environment. The results demonstrate that the 
Rust implementation in \eng{Zero-Copy} mode fully saturates the physical bandwidth. Of particular note is the resource efficiency: 
SYN-Rust processed approximately three times more packets per percent of CPU utilization than Masscan and four times more than ZMap. 
Thus, this thesis demonstrates that Rust, in combination with modern kernel mechanisms, represents a powerful and secure alternative 
to C for the development of system-level network scanners.
% !TEX root = ../Thesis.tex
%%
%%  Hochschule für Technik und Wirtschaft Berlin --  Abschlussarbeit
%%
%%  Abstract - Englisch
%%
%%%%%%%%%%%%%%%%%%%%%%%%%%%%%%%%%%%%%%%%%%%%%%%%%%%%


\section*{Abstract}

Horizontal network scanning is a fundamental method in proactive security research. Established high-performance scanners 
such as ZMap or Masscan are predominantly based on C, which enables maximum performance but poses risks of security 
vulnerabilities due to a lack of memory safety. This thesis investigates the extent to which a scanner implemented in 
Rust can compete with these tools in terms of throughput and resource efficiency, while offering an intrinsically higher 
level of security through the language's inherent guarantees.

To this end, an asynchronous \texttt{SYN} scanner (\enquote{SYN-Rust}) was developed that leverages modern Linux kernel interfaces such as 
\texttt{AF\_XDP} and \texttt{eBPF} to partially bypass the network stack. Complementing this, the user-space employs a logically decoupled 
architecture utilizing the \texttt{tokio} runtime to ensure efficient concurrency. The prototype was evaluated against ZMap and 
Masscan in a controlled Gigabit test environment.

The results indicate that the Rust implementation in zero-copy mode fully saturates the bandwidth limit of the Gigabit 
interface. Particularly noteworthy is the resource efficiency: SYN-Rust processed approximately three times more packets
per percentage point of CPU utilization than Masscan and four times more than ZMap. The thesis thus demonstrates that Rust, 
combined with modern kernel mechanisms, represents a powerful and secure alternative to C for developing low-level network 
applications in the field of network scanning.

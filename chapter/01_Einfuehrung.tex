% !TEX root = ../Thesis.tex
%%
%%  Hochschule für Technik und Wirtschaft Berlin --  Abschlussarbeit
%%
%% Kapitel 1
%%
%%

\chapter{Einleitung} \label{Einleitung}

\section{Motivation und Einführung in das Themengebiet} \label{Einführung}

Netzwerk-Scanning macht einen großen Teil des Internet-Verkehrs im IPv4 Adressraumes aus. So ist 98 Prozent des gesamten TCP Verkehrs weltweit auf \texttt{SYN}-Scans zurückzuführen \cite{griffioen2024have}. 
Bekannte Tools wie \textit{ZMap} werden stetig weiterentwickelt \cite{Durumeric_Adrian_Stephens_Wustrow_Halderman_2024} und sind seit der Entwicklung von performanten Open-Source Scannern wie \textit{ZMap} 
oder \textit{Masscan} \cite{Durumeric_Wustrow_Halderman}\cite{Graham_2026} dazu fähig,
den gesamten IPv4 Adressraum in weniger als 45min zu scannen. Das Scannen von Netzwerken nach offenen Ports ermöglicht es Organisationen Schwachstellen ausfindig zu machen, 
bevor Angreifer es tun. Außerdem lassen sich durch das breitflächige Scannen von ausgewählten Adressräumen oder dem gesamten IPv4 Raum Informationen über Trends und Veränderungen 
dieser ableiten. Cyberangriffe haben Auswirkungen auf den Ruf und die finanzielle Stabilität von Unternehmen \cite{Rudnev_Zolkin_Artemyev_Tychkov_2024}. Die gegenwärtig hohen Angriffszahlen zum Beispiel bei 
\textit{Denial-of-Service} Angriffen \cite{Falowo_Okpala_Kojo_Azumah_Li_2023} unterstreichen die Wichtigkeit.

Bisherige Hochleistungsscanner, wie die soeben genannten, wurden überwiegend in C entwickelt \cite{Durumeric_Wustrow_Halderman}\cite{Graham_2026}\cite{Li_2026}\cite{Li_Zhang_Guo_Bao_Xu_Hu_Li_2022}. 
C ist häufig die Standardwahl für maschinennahe Anwendungen, da sie zum einen ein 
niedriges Level an Abstraktion und zum anderen hochperformant sein kann [TODO]. Allerdings ist C anfällig für menschengemachte Fehler \cite{Al_Boghdady_Wassif_El_Ramly_2021} wie ..TODO.. [TODO], von welchen
die meisten sicherheitsrelevanten aus der Fraktion der Speicherverwaltung stammen \cite{bugden2022safety}. Andere Sprachen wie Go oder Python lösen einige dieser Probleme durch die Nutzung einer 
automatischen Speicherverwaltung [TODO]. Diese Sprachen sind allerdings im Vergleich zu Sprachen wie C weniger performant \cite{bugden2022safety}. 

Rust hingegen schneidet in Vergleichen bezüglich der Performance auf ähnlichem Niveau wie C ab, bringt gleichzeitig aber das höchste Sicherheitsniveau der genannten 
Sprachen mit \cite{bugden2022safety}\cite{Costanzo_Rucci_Naiouf_Giusti_2021}. Außerdem unterstützt Rust Konzepte von Sprachen hoher Abstraktionsebene, wie beispielsweise die der funktionalen 
Programmierung oder Objektorientierung \cite{Costanzo_Rucci_Naiouf_Giusti_2021}, während zudem in der
zuletzt zitierten Untersuchung, auch die Anzahl der Zeilen niedriger als im Vergleich zu dem in C geschriebenen Code ist.

Bisher fehlt eine fundierte Untersuchung darüber, ob Rust als moderne Sprache, welche Sicherheitsgarantien, \textit{high-level} \footnote{Auf hoher Abstraktionsebene} Konzepte und Performance vereint,
in Kombination mit aktuellen Linux-Schnittstellen, in der Lage ist, eine konkurrenzfähige Alternative zu gängigen Hochleistungsscannern, welche überwiegend in C geschrieben sind, darzustellen.
Es ist ungeklärt, ob der potenzielle Performanceunterschied gering genug ist, um durch die gewonnene Sicherheit kompensiert zu werden, weshalb
diese Arbeit an diesem Punkt ansetzt.

\section{Zielsetzung und Forschungsfrage} \label{Zielsetzung}

In dieser Arbeit wird ein prototypischer \texttt{SYN}-Portscanner zum breitflächigen Scannen von Netzwerken in Rust entwickelt. 
Der Fokus des Scanners liegt auf einer hohen Performance, weshalb die Architektur teilweise asynchron gestaltet und leistungsfähige Linux-Schnittstellen 
wie \texttt{AF\_PACKET} und \texttt{XDP} verwendet werden. Anschließend wird dieser bezüglich ausgewählter Performance Metriken
mit einer repräsentativen Auswahl an bestehenden Scannern verglichen und die Ergebnisse daraufhin evaluiert.

Es ergibt sich folgende Forschungsfrage: Inwieweit kann ein in Rust implementierter asynchroner \texttt{SYN}-Scanner hinsichtlich des Durchsatzes und der Ressourceneffizienz mit etablierten Hochleistungsscannern konkurrieren und durch spracheigene Sicherheitsgarantien eine tragfähige Alternative für den produktiven Einsatz darstellen?

\section{Abgrenzung des Themas} \label{Abgrenzung}
%The best port scanning technique is TCP SYN scan as it has the lowest response time and thus the least impact on the target host.
%TCP SYN scan is also well-known as the stealthiest port scanning technique. cite A Comparative Performance of Port Scanning Techniques
% -> Es geht nur um SYN Scan weil ...

Bei der in dieser Arbeit entwickelten Implementierung handelt es sich um einen horizontalen Scanner \ref{Grundlagen}.
Anders als beispielsweise beim regulären \texttt{SYN}-Scan des Tools \textit{Nmap} \cite{Lyon_2010}, welcher in der Regel vertikal erfolgt.
% Begründen (gibt explizit Punkte)

Zusätzliche Mechanismen zur Verschleierung des Scans oder weiterführende Maßnahmen zur Treffererhöhung
werden in dieser Implementierung rudimentär oder gar nicht behandelt, da der Fokus
auf der Nutzung von Rust, sowie der Entwicklung eines Performanz-orientieren Netzwerkscanners liegt. 
Da der normale Ablauf des SYN-Scans bereits grundlegende Mechanismen in den Bereichen mitbringt \cite{Universiti_Tun_Hussein_Onn_Malaysia_Roslan_2023}, 
sind diese Gebiete für die Beantwortung der Forschungsfrage nicht notwendig. Die weiterführende 
Untersuchung der Ergebnisse bildet einen eigenen Forschungszweig und ist somit auch ausgeschlossen.

Außerdem beschränkt sich diese Arbeit auf den IPv4 Adressraum, da dies genügt, um der Forschungsfrage nachzugehen.
% oder maybe da es eine prototypische Implementierung ist oder so idk






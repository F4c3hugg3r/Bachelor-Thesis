% !TEX root = ../Thesis.tex
%%
%%  Hochschule für Technik und Wirtschaft Berlin --  Abschlussarbeit
%%
%% Kapitel 1
%%
%%

% Referenzen korrekt setzen -> Falls eine Referenz nicht gülit ist, oder TODO dran steht, Referenz am 

% Checkliste Quellen und Zitation
% Bei mehrfachzitierung der gleichen Quelle folgendes beachten:
% Bei 3 Sätzen einer Quelle nur angemessen am Ende oder nach dem ersten Satz Verweis einfügen 
% -> hinter Punkt, wenn ganzer Absatz einer Quelle angehört: .[1]; vor dem Punkt, wenn es nur für den Satz gilt: [1].
% Bei Quellen z.B. Konferenzartikel / Buch mit vielen Seiten, die Seitenzahl des Berichts / der Information dazu schreiben

% Checkliste Inhalt
% immer kurzen Absatz nach einer Überschrift

% Außerdem: 
% Diagramme, alle sachen deutsch - muss nicht
% (Direktzitieren lieber komplett vermeiden)
% (Eigenständigkeitserklärung für KI auch reinnehmen) 
% Kontext Arbeit -> wie begründen, dass mock_program genutzt wird: durch den betrieblichen Rahmen gegeben Grundvorasussetzung und dann im Aussichtsteil  
% wann kolloquium -> dauert bestimmt erst Noten eingetragen werden 


% TODO Grafiken als PDF statt png importieren 

\chapter{Einleitung} \label{Einleitung}

\section{Motivation und Einführung in das Themengebiet} \label{Einführung}

Das Netzwerk-Scanning macht einen nicht zu vernachlässigenden Teil des Internet-Verkehrs im
IPv4-Adressraum aus. So waren 98 Prozent des unaufgeforderten TCP-Verkehrs im Jahre
2024 weltweit auf \texttt{SYN}-Scans zurückzuführen \cite{griffioen2024have}. 
Bekannte \eng{Open-Source}-Internet-Scanning-Projekte wie ZMap, welches über 10 Jahre stetig weiterentwickelt wurde \cite{Durumeric_Adrian_Stephens_Wustrow_Halderman_2024} oder Masscan \cite{Graham_2026} 
sind dazu in der Lage \cite{Durumeric_Wustrow_Halderman} den gesamten IPv4-Adressraum in der Größenordnung von Minuten zu scannen. Das Scannen von Netzwerken nach offenen Ports ermöglicht es Organisationen Schwachstellen ausfindig zu machen, 
bevor Angreifer es tun. Außerdem lassen sich durch das breitflächige Scannen von ausgewählten Adressräumen oder dem gesamten IPv4-Raum Informationen über Trends und Veränderungen 
dieser ableiten. Cyberangriffe haben Auswirkungen auf den Ruf und die finanzielle Stabilität von Unternehmen \cite{Rudnev_Zolkin_Artemyev_Tychkov_2024}. Die gegenwärtig hohen Angriffszahlen zum Beispiel bei 
\eng{Denial-of-Service}-Angriffen \cite{Falowo_Okpala_Kojo_Azumah_Li_2023} unterstreichen die Wichtigkeit.

Bisherige Hochleistungsscanner, wie die soeben genannten, wurden überwiegend in C entwickelt \cite{Durumeric_Wustrow_Halderman}\cite{Graham_2026}\cite{Li_2026}\cite{Li_Zhang_Guo_Bao_Xu_Hu_Li_2022}. 
C ist häufig die Standardwahl für maschinennahe Anwendungen, da sie zum einen ein 
niedriges Level an Abstraktion und zum anderen hochperformant sein kann \cite{Peta_2022}. Allerdings ist C anfällig für menschengemachte Fehler \cite{Al_Boghdady_Wassif_El_Ramly_2021} 
wie doppelte Speicher-Freigaben, Zugriffe auf bereits freigegebenen Speicher und Pufferüberläufe welche teils zu Speicherbeschädigungen und Sicherheitslücken führen können \cite{bugden2022safety} \cite{van_Oorschot_2023}. 
Andere Sprachen wie Go oder Python lösen einige dieser Probleme durch die Nutzung einer lösen viele Speicherprobleme durch automatische Speicherverwaltung, insbesondere durch \eng{Garbage Collection} 
und andere Techniken. Diese Sprachen sind allerdings im Vergleich zu Sprachen ohne automatischer Speicherverwaltung wie C weniger performant \cite{bugden2022safety}. 

Rust hingegen schneidet in Vergleichen bezüglich der \eng{Performance} auf ähnlichem Niveau wie C ab, bringt gleichzeitig aber das höchste Sicherheitsniveau der genannten 
Sprachen mit, indem es Speicherfehler weitestgehend verhindert \cite{bugden2022safety}\cite{Costanzo_Rucci_Naiouf_Giusti_2021}. Außerdem unterstützt Rust Konzepte von Sprachen hoher Abstraktionsebene, wie beispielsweise die der funktionalen 
Programmierung oder Objektorientierung \cite{Costanzo_Rucci_Naiouf_Giusti_2021}, während zudem in der
zuletzt zitierten Untersuchung, auch die Anzahl der Zeilen niedriger als im Vergleich zu dem in C geschriebenen Code ist.

Bisher fehlt eine fundierte Untersuchung darüber, ob Rust als moderne Sprache, welche Sicherheitsgarantien, \eng{High-Level}\footnote{Auf hoher Abstraktionsebene} Konzepte und \eng{Performance} vereint,
in Kombination mit aktuellen Linux-Schnittstellen, in der Lage ist, eine konkurrenzfähige Alternative zu gängigen Hochleistungsscannern, welche überwiegend in C geschrieben sind, darzustellen.
Es ist ungeklärt, ob der potenzielle \eng{Performance}-Unterschied gering genug ist, um durch die gewonnene Sicherheit kompensiert zu werden, weshalb
diese Arbeit an diesem Punkt ansetzt.

\section{Zielsetzung und Forschungsfrage} \label{Zielsetzung}

In dieser Arbeit wird ein prototypischer \texttt{SYN}-Portscanner zum breitflächigen Scannen von Netzwerken in Rust entwickelt. 
Der Fokus des Scanners liegt auf einer hohen \eng{Performance} sowie hohen Effizienz, weshalb die Architektur teilweise asynchron gestaltet und leistungsfähige Linux-Schnittstellen 
wie \texttt{AF\_PACKET}, \texttt{AF\_XDP} und \texttt{eBPF} verwendet werden. Anschließend wird dieser bezüglich ausgewählter \eng{Performance}-Metriken
mit einer repräsentativen Auswahl an bestehenden Scannern verglichen und die Ergebnisse daraufhin evaluiert.

Es ergibt sich folgende Forschungsfrage: Inwieweit kann ein in Rust implementierter asynchroner \texttt{SYN}-Scanner hinsichtlich des Durchsatzes und der 
Ressourceneffizienz mit etablierten Hochleistungsscannern konkurrieren und durch sprach-eigene Sicherheitsgarantien eine tragfähige Alternative für den 
produktiven Einsatz darstellen?

\section{Abgrenzung des Themas} \label{Abgrenzung}
Die Scanning-Methode beschränkt sich explizit auf das \texttt{SYN}-Scanning. Es ist die de facto Standard Methode und 
weist in Tests den niedrigsten Einfluss auf das Zielsystem, sowie die kürzeste Scan-Dauer auf \cite{Universiti_Tun_Hussein_Onn_Malaysia_Roslan_2023}. 

Bei der in dieser Arbeit entwickelten Implementierung handelt es sich um einen horizontalen Scanner \ref{Grundlagen}.
Anders als beispielsweise beim regulären \texttt{SYN}-Scan des Tools Nmap \cite{Lyon_2010}, welcher in der Regel vertikal erfolgt.
Vertikales Scanning ist für Netzwerk- beziehungsweise Internetscanner weniger relevant, da dabei das individuelle Ziel 
im Vordergrund steht. (TODO muss das belegt werden? Falls ja Nmap)

Zusätzliche Mechanismen zur Verschleierung des Scans oder weiterführende Maßnahmen zur Treffererhöhung
werden in dieser Implementierung rudimentär oder gar nicht behandelt, da der Fokus
auf der Nutzung von Rust, sowie der Entwicklung eines \eng{Performance}-orientieren Netzwerkscanners liegt. 
Da der normale Ablauf des \texttt{SYN}-Scans bereits grundlegende Mechanismen in den Bereichen mitbringt \cite{Universiti_Tun_Hussein_Onn_Malaysia_Roslan_2023}, 
sind diese Gebiete für die Beantwortung der Forschungsfrage nicht notwendig. Die weiterführende 
Untersuchung der Ergebnisse bildet einen eigenen Forschungszweig und ist somit auch ausgeschlossen.

Außerdem beschränkt sich diese Arbeit auf den IPv4-Adressraum, da dies genügt, um der Forschungsfrage nachzugehen.
% oder maybe da es eine prototypische Implementierung ist oder so idk






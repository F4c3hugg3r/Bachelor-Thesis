% !TEX root = ../Thesis.tex
%%
%%  Hochschule für Technik und Wirtschaft Berlin --  Abschlussarbeit
%%
%% Kapitel 1
%%
%%

% Checkliste Quellen und Zitation
% Bei mehrfachzitierung der gleichen Quelle folgendes beachten:
% Bei 3 Sätzen einer Quelle nur angemessen am Ende oder nach dem ersten Satz Verweis einfügen 
% -> hinter Punkt, wenn ganzer Absatz einer Quelle angehört: .[1]; vor dem Punkt, wenn es nur für den Satz gilt: [1].
% Bei Quellen z.B. Konferenzartikel / Buch mit vielen Seiten, die Seitenzahl des Berichts / der Information dazu schreiben

% Checkliste Inhalt
% immer kurzen Absatz nach einer Überschrift

% Leerzeichen vor cites entfernen
\chapter{Einleitung} \label{Einleitung}

\section{Motivation und Einführung in das Themengebiet} \label{Einführung}
Das Scannen von Netzwerken oder gar dem gesamten Internet macht einen nicht zu vernachlässigenden Teil des Datenverkehrs im IPv4-Adressraum aus. 
So waren im Jahr 2024 weltweit 98 Prozent des unaufgeforderten TCP-Verkehrs auf \texttt{SYN}-Scans zurückzuführen\cite{griffioen2024have}. 
Etablierte \eng{Open-Source}-Projekte wie ZMap\cite{Durumeric_Adrian_Stephens_Wustrow_Halderman_2024} oder Masscan\cite{Graham_2026} 
sind in der Lage, den gesamten IPv4-Adressraum innerhalb weniger Minuten zu scannen\cite{Durumeric_Wustrow_Halderman_182948}.
Durch das proaktive Scannen eigener Netzwerke können Schwachstellen identifiziert werden, bevor diese von Angreifern ausgenutzt werden. 
Darüber hinaus liefern breit angelegte Scans empirische Daten über globale Trends und Veränderungen 
in der Sicherheitslandschaft und stellen somit eine Datengrundlage für die Sicherheitsforschung dar. Angesichts der Tatsache, 
dass Cyberangriffe, wie beispielsweise \eng{Denial-of-Service}-Attacken\cite{Falowo_Okpala_Kojo_Azumah_Li_2023}, sowohl die 
Reputation als auch die finanzielle Stabilität von Unternehmen massiv gefährden\cite{Rudnev_Zolkin_Artemyev_Tychkov_2024}, ist 
die Verfügbarkeit und Weiterentwicklung leistungsfähiger Analysewerkzeuge von kritischer Bedeutung.

Bisherige Hochleistungsscanner wurden überwiegend in C entwickelt\cite{Durumeric_Wustrow_Halderman_182948, Graham_2026, Li_2026, Li_Zhang_Guo_Bao_Xu_Hu_Li_2022}. 
C ist häufig die Standardwahl für maschinennahe Anwendungen, da sie zum einen ein 
niedriges Abstraktionsniveau und zum anderen hochperformant sein kann\cite{Peta_2022}. Allerdings ist C anfällig für menschengemachte Fehler\cite{Al_Boghdady_Wassif_El_Ramly_2021} 
wie doppelte Speicherfreigaben, Zugriffe auf bereits freigegebenen Speicher und Pufferüberläufe, welche teils zu Speicherbeschädigungen und Sicherheitslücken führen können\cite{bugden2022safety, van_Oorschot_2023}. 
Andere Sprachen wie zum Beispiel Go, Java oder Python lösen diese Probleme durch automatische Speicherverwaltung, insbesondere durch \eng{Garbage Collection} 
und weitere Techniken. Diese Sprachen sind allerdings im Vergleich zu Sprachen ohne automatische Speicherverwaltung wie C weniger performant\cite{bugden2022safety}. 

Rust hingegen schneidet in Vergleichen bezüglich der \eng{Performance} auf ähnlichem Niveau wie C ab, bringt gleichzeitig aber das höchste Sicherheitsniveau der genannten 
Sprachen mit, indem es Speicherfehler weitestgehend verhindert\cite{bugden2022safety, Costanzo_Rucci_Naiouf_Giusti_2021}. Außerdem unterstützt Rust Konzepte von 
Sprachen hoher Abstraktionsebene, wie beispielsweise die der funktionalen 
Programmierung oder Objektorientierung\cite{Costanzo_Rucci_Naiouf_Giusti_2021}, wobei in der genannten Untersuchung zudem der 
Codeumfang geringer ausfiel als bei der untersuchten C-Variante.

Bisher fehlt eine fundierte Untersuchung darüber, ob Rust als moderne Sprache, welche Sicherheitsgarantien, \eng{High-Level}\footnote{Auf hoher Abstraktionsebene} Konzepte und \eng{Performance} vereint,
in Kombination mit aktuellen Linux-Schnittstellen wie \texttt{AF\_XDP} oder \texttt{eBPF} in der Lage ist, eine konkurrenzfähige Alternative zu gängigen Hochleistungsscannern, welche überwiegend in C geschrieben sind, darzustellen.
Es ist ungeklärt, ob der potenzielle \eng{Performance}-Unterschied gering genug ist, um durch die gewonnene Sicherheit kompensiert zu werden, weshalb
diese Arbeit an diesem Punkt ansetzt.

\section{Zielsetzung und Forschungsfrage} \label{Zielsetzung}

In dieser Arbeit wird ein prototypischer \texttt{SYN}-Portscanner zum breitflächigen Scannen von Netzwerken in Rust entwickelt. 
Der Fokus des Scanners liegt auf einer hohen \eng{Performance} sowie hoher Effizienz, weshalb die Architektur teilweise asynchron gestaltet wird und leistungsfähige Linux-Schnittstellen 
wie \texttt{AF\_PACKET}, \texttt{AF\_XDP} und \texttt{eBPF} verwendet werden. Anschließend wird dieser bezüglich ausgewählter \eng{Performance}-Metriken
mit einer repräsentativen Auswahl an bestehenden Scannern verglichen und die Ergebnisse daraufhin evaluiert.

Es ergibt sich folgende Forschungsfrage: Inwieweit kann ein in Rust implementierter asynchroner \texttt{SYN}-Scanner hinsichtlich des Durchsatzes und der 
Ressourceneffizienz mit etablierten Hochleistungsscannern konkurrieren und durch sprach-eigene Sicherheitsgarantien eine tragfähige Alternative für den 
produktiven Einsatz darstellen?

\section{Abgrenzung des Themas} \label{Abgrenzung}
Die Scanning-Methode beschränkt sich explizit auf das \texttt{SYN}-Scanning. Es ist die \eng{de facto} Standardmethode und 
weist in Tests den niedrigsten Einfluss auf das Zielsystem sowie die kürzeste Scan-Dauer auf\cite{Universiti_Tun_Hussein_Onn_Malaysia_Roslan_2023}. 

Bei der in dieser Arbeit entwickelten Implementierung handelt es sich um einen horizontalen Scanner (siehe \cref{Portscanning}),
anders als beispielsweise beim regulären \texttt{SYN}-Scan des Tools Nmap\cite{Lyon_2010}, welcher in der Regel vertikal erfolgt.
Vertikales Scanning ist für Netzwerk- beziehungsweise Internetscanner weniger relevant, da dabei das individuelle Ziel 
im Vordergrund steht. 

Zusätzliche Mechanismen zur Verschleierung des Scans oder weiterführende Maßnahmen zur Treffererhöhung
werden in dieser Implementierung lediglich rudimentär behandelt, da der Fokus
auf der Nutzung von Rust sowie der Entwicklung eines \eng{Performance}-orientieren Netzwerkscanners liegt. 
Da der normale Ablauf des \texttt{SYN}-Scans bereits grundlegende Mechanismen diesbezüglich mitbringt\cite{Universiti_Tun_Hussein_Onn_Malaysia_Roslan_2023}, 
sind diese Gebiete für die Beantwortung der Forschungsfrage nicht notwendig.
Außerdem beschränkt sich diese Arbeit auf den IPv4-Adressraum, da dies genügt, um der Forschungsfrage nachzugehen.






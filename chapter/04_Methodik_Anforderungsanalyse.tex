% !TEX root = ../Thesis.tex
%%
%%  Hochschule für Technik und Wirtschaft Berlin --  Abschlussarbeit
%%
%% Kapitel 4 - Konzeption und Implementierung
%%
%%

\chapter{Methodik und Anforderungsanalyse} \label{Methodik}
Dieses Kapitel definiert die funktionalen und nicht-funktionalen Anforderungen an den zu entwickelnden Portscanner, 
beschreibt das gewählte Vorgehensmodell zur Umsetzung in Rust und legt das Untersuchungsdesign für die anschließende Evaluation fest.

\section{Anforderungsanalyse}

\subsection{Funktionale Anforderungen}
-	Pakete versenden in hoher Geschwindigkeit mind. 5stellige pps
-	Pakete empfangen in hoher Geschwindigkeit
-	Stateless
-	Pakete korrekt bauen und antworten korrekt interpretieren
-	Mglw. RST senden

\subsection{Nicht-funktionale Anforderungen}


\section{Vorgehensmodell der Entwicklung}
Für die Realisierung wird ein evaluationsgetriebener, prototypischer Ansatz gewählt. Aufgrund der Komplexität asynchroner 
Netzwerkprogrammierung, sowie der Vielzahl möglicher Technologien, empfiehlt es sich, verschiedene Wege auszuprobieren. Dies
hilft bei der Schaffung einer realistischen Vergleichsbasis zwischen den verschiedenen Technologien und bietet die Möglichkeit,
Ausweichlösungen bei Problemen wie z.B. Performance-Engpässe zu finden oder die tatsächliche Notwendigkeit komplexer 
Technologien zu validieren. Die Entwicklung teilt sich in 2 Hauptphasen:

\begin{enumerate}
    \item \textbf{Basisimplementierung:} In der Basisimplementierung wird ein vollumfänglicher SYN-Scanner implementiert.
    Dies soll zum einen als \textit{Proof of Concept}, und zum anderen als erste Vergleichsgrundlage dienen. Für 
    das Senden, werden bereits die Möglichkeit des Sendens mit AF\_PACKET oder AF\_XDP und jeweils auch in \textit{Batches}\footnote{Gruppen von Paketen}
    oder Einzelpaketen implementiert. Für das Empfangen wird der \textit{Crate}\footnote{Bezeichnung für Bibliothek im Rust-Ökosystem} \textit{pcap} genutzt,
    welcher das Empfangen von Paketen abstrahiert, allerdings nicht den Netzwerkstack des Linux-Kernels umgeht. 

    Außerdem wird die für die nötige Struktur um die Komponenten zu vernetzen, sowie die Pakete korrekt zu erstellen
    und verarbeiten implementiert, sodass das Programm am Ende einen funktionierenden SYN-Scan vollführen kann.
    Zusätzlich wird auch hier schon auf die möglichst performante Umsetzung der gesamten Struktur, spezifischer
    Umsetzungen und der Wahl von Crates Wert gelegt. 

    \item \textbf{Optimierung des Empfangspfades durch Nutzung von eBPF:} Basierend auf den Messergebnissen der ersten Phase 
    wird die Empfangskomponente hier grundsätzlich verändert, um eine hochperformante, sowie effizienten Paketempfang und
    Paketauswertung zu gewährleisten. Dafür wird ein eBPF in Verbindung mit einem RingBuf zur Protokollierung in Verbindung
    mit einem XDP Programm statt des pcap-Crates genutzt. maybe TODO "bla bla pcap kann ab gewisser Geschwindigkeit nicht mithalten
    oder pcap ist zu ineffizient und/oder rst werden immer automatisch gesendet". Außerdem werden weitere, kleinere
    Performanz-steigernde Maßnahmen addressiert.
\end{enumerate}


\section{Untersuchungsdesign}
Hier müssen Sie definieren, wie Sie vergleichen.
-	Definition der Metriken: Warum messen Sie pps? Warum RAM? Wie messen Sie CPU (User vs. System Time)?
-	Definition der Testumgebung: (Hier nur theoretisch – z.B. "Isolierte Laborumgebung um Netzwerrauschen zu minimieren").



% !TEX root = ../Thesis.tex
%%
%%  Hochschule für Technik und Wirtschaft Berlin --  Abschlussarbeit
%%
%% Kapitel 4 - Konzeption und Implementierung
%%
%%

\chapter{Anforderungsanalyse und Methodik} \label{Methodik}
Dieses Kapitel definiert die funktionalen und nicht-funktionalen Anforderungen an den zu entwickelnden Portscanner, 
beschreibt das gewählte Vorgehensmodell zur Umsetzung in Rust und legt das Untersuchungsdesign für die anschließende Evaluation fest.

\section{Anforderungsanalyse} \label{Anforderungen}

\subsection{Funktionale Anforderungen} \label{Anforderungen.fa}
Die funktionalen Anforderungen definieren das Verhalten des Systems und die logischen Operationen, die der Scanner ausführen muss, 
um einen korrekten \texttt{SYN}-Scan durchzuführen.

\begin{itemize} 
    \item \textbf{/F-01/ \phantomsection\label{req:F-01} Konstruktion valider TCP-\texttt{SYN}-Pakete:} Das System muss in der Lage sein, rohe TCP-Pakete so zu konstruieren, 
    dass \eng{IP}-Header und \eng{TCP}-Header (inklusive \texttt{SYN}-Cookie) korrekt manuell gesetzt und die Prüfsummen valide berechnet werden,
    damit sie vom Zielsystem als legitime Verbindungsanfragen akzeptiert werden. 

    \item \textbf{/F-02/ \phantomsection\label{req:F-02} Senden von Paketen:} Das System muss in der Lage sein, die konstruierten TCP-Pakete über die Netzwerkschnittstelle an definierte Zielsysteme zu versenden. 
    
    \item \textbf{/F-03/ \phantomsection\label{req:F-03} Empfang von Paketen:} Das System muss in der Lage sein, eingehende Netzwerkpakete unabhängig vom Sendeprozess abzufangen und zur Auswertung bereitzustellen.

    \item \textbf{/F-04/ \phantomsection\label{req:F-04} Zustandsloses Scanning:} Die Sende- und Empfangskomponenten dürfen keine statusbehaftete Kommunikation über die Zielsysteme austauschen. 
    Die Zuordnung muss ausschließlich über Informationen im Paket-Header erfolgen.

    \item \textbf{/F-05/ \phantomsection\label{req:F-05} Validierung eingehender Antworten:} Die Empfangskomponente muss eingehende \texttt{SYN-ACK}-Pakete validieren. 
    Dafür muss der Hash-Werte des \texttt{SYN}-Cookies korrekt erstellt und mit dem aus der \eng{Acknowledgement Number} extrahierten Wert verglichen werden.

    \item \textbf{/F-06/ \phantomsection\label{req:F-06} Schließen der Verbindung:} Nach der Identifikation eines offenen Ports muss der Scanner 
    ein \texttt{RST}-Paket senden, um die halboffene-Verbindung auf dem Zielsystem sauber zu beenden.

    \item \textbf{/F-07/ \phantomsection\label{req:F-07} Endausgabe:} Es muss eine Endausgabe in einer Datei oder dem \eng{Standard Output} 
    geben in welcher die ausgewerteten Scanergebnisse bestehend aus IP-Adresse und Ziel-Port der offenen Zielsysteme enthalten sind.

    \item \textbf{/F-08/ \phantomsection\label{req:F-08} Durchsatzlimitierung:} Das Programm muss in der Lage sein, eine angegebene Durchsatzrate (in Byte pro Sekunde)
    nicht zu überschreiten, sodass eine konsistente \eng{Performance}-Messung möglich ist.

    \item \textbf{/F-09/ \phantomsection\label{req:F-09} Eingabeschnittstelle:} Das Programm muss die zu scannenden Ziel-IP-Adressen aus dem \eng{Standard Input}
    des Programmes entnehmen, um in die Infrastruktur des Unternehmens, welches diese Arbeit begleitet, zu passen.
\end{itemize}

\subsection{Nicht-funktionale Anforderungen}
Die nicht-funktionalen Anforderungen stellen Qualitätsanforderungen dar und leiten sich primär aus technischen Randbedingungen
und der Verwendung von Rust ab, welche sich aus dem Forschungsziel ergeben.

\begin{itemize} 
\item \textbf{/NF-01/ \phantomsection\label{req:NF-01} Maximierung des Durchsatzes:} Das System soll in der Lage sein, die verfügbare Bandbreite einer 
Standard-Gigabit-Schnittstelle vollständig auszunutzen.

    \item \textbf{/NF-02/ \phantomsection\label{req:NF-02} Asynchrone Architektur:} Die Implementierung muss auf einem asynchronen Programmiermodell basieren, 
    um durch nicht-blockierende I/O-Operationen eine hohe Nebenläufigkeit zu gewährleisten.

    \item \textbf{/NF-03/ \phantomsection\label{req:NF-03} Nutzung moderner Kernel-Mechanismen:} 
    Zur Evaluation der Forschungsfrage müssen Linux-native Schnittstellen zur hochperformanten Paketverarbeitung wie \texttt{AF\_XDP} 
    oder \texttt{eBPF} verwendet werden.

\item \textbf{/NF-04/ \phantomsection\label{req:NF-04} Speichersicherheit:} 
Die Implementierung soll die Sicherheitsgarantien von Rust wahren. \texttt{unsafe}-Blöcke können genutzt werden 
(z.B. in der Interaktion mit Kernel-APIs) sollten aber möglichst vermieden oder durch die Nutzung von externen Bibliotheken  
ersetzt werden, da diese intern \texttt{unsafe}-Blöcke häufig sicher kapseln \cite{RustBelt}.

\item \textbf{/NF-05/ \phantomsection\label{req:NF-05} Minimale Ressourcennutzung:} Der CPU- und Arbeitsspeicherverbrauch soll im Verhältnis zum 
erzielten Durchsatz minimiert werden.

 \item \textbf{/NF-06/ \phantomsection\label{req:NF-06} Technologische Einschränkung:} Das Programm darf ausschließlich
 Technologien verwenden, die im Linux-Kernel-Ökosystem verfügbar sind, um Abhängigkeiten von Drittanbieter-Treibern zu vermeiden.

 \end{itemize}

% TODO eventuell auf einheitliche Formulierung achten -> bishe manchmal "sollen ..." manchmal "wird ..."
% TODO formatting 
\section{Untersuchungsdesign}
In dieser Sektion wird das methodische Vorgehen zur Validierung der Anforderungen beschrieben. 
Hierbei wird zwischen der dynamischen Überprüfung durch Tests und der statischen Verifikation durch 
Inspektion des Designs unterschieden.

\subsection{Nachweismethoden}
Die Verifikation der in \cref{Anforderungen} definierten Anforderungen erfolgt anhand von zwei Methoden:

\begin{itemize}
    \item \textbf{Dynamische Tests:} Diese validieren das Laufzeitverhalten und die Performanz des Systems. 
    Anforderungen wie das korrekte Senden und Empfangen von Paketen 
    (\cref{req:F-01,req:F-02,req:F-03,req:F-05,req:F-06,req:F-07}) oder die Einhaltung eines 
    Durchsatzlimits (\cref{req:F-08}) werden durch explizite Testfälle (\eng{Proof of Concept}) und 
    Evaluationsszenarien nachgewiesen.
    
    \item \textbf{Statische Inspektion:} Anforderungen, die sich auf die Architektur, die Wahl der 
    Programmiersprache oder die Verwendung spezifischer Kernel-Schnittstellen beziehen, 
    werden durch die Inspektion der Implementierung verifiziert. Der Nachweis für die asynchrone 
    Architektur (\cref{req:NF-02}), die Nutzung von \texttt{AF\_XDP} und \texttt{eBPF} (\cref{req:NF-03}),
    die Speichersicherheit durch Rust (\cref{req:NF-04}), die technologische Einschränkung (\cref{req:NF-06}) 
    oder der generelle Aufbau (\cref{req:F-04,req:F-09}) gilt als erbracht, indem die entsprechenden Konzepte 
    im Design verankert und im Quellcode umgesetzt wurden (siehe \cref{Implementierung}).
\end{itemize}

Anschließend wird erklärt, wie \eng{Performance} im Kontext eines \texttt{SYN}-Scanners zu definieren ist und zuletzt werden die, in dieser Arbeit zur Evaluation genutzten, Metriken und Evaluationsszenarien festgelegt.

\subsection{Evaluationstests für den \eng{Proof of Concept}}
Um die Funktionsweise, beziehungsweise den Scanner nach \cref{Anforderungen} prüfen zu können,
werden zwei Tests durchgeführt:

\begin{enumerate}
    \crefalias{enumi}{test}
    \item \textbf{/T-01/\phantomsection\label{req:T-01} Sende-und Empfangsvalidierung:} Der erste Test dient als Validierung der Anforderungen \cref{req:F-02,req:F-03,req:F-06,req:F-07}. 
    Es werden Pakete verschickt und von einer anderen, antwortenden Instanz empfangen. Dabei wird untersucht, ob die korrekte Anzahl an SYN-Paketen verschickt, 
    SYN-ACK-Paketen empfangen und RST-Antworten verschickt wird\footnote{Mit SYN / SYN-ACK / RST ist der gesetzte Wert der TCP-Flags (siehe \cref{tab:TCP-header-fields}) gemeint. }.
    Zur Validierung werden die Werte der Ausgaben des Scanner-Knotens mit den empfangenen Paketen des Ziel-Knotens und den erwarteten Werten verglichen.
    \crefalias{enumi}{test}
    \item \textbf{/T-02/\phantomsection\label{req:T-02} Paketvalidierung:} Im zweiten Test wird die Korrektheit der erstellten Pakete validiert, um die 
    Umsetzung der Anforderungen \cref{req:F-01,req:F-05} zu überprüfen. Dafür werden Pakete an eine andere, antwortende Instanz verschickt. Diese antwortet nur auf korrekte Pakete und 
    stoppt das Senden von Antworten, wenn valide RST-Pakete eingehen. In dem Test wird somit untersucht, ob diese Verhaltensweisen auftreten. Zusätzlich wird 
    mithilfe von externen Tools  
\end{enumerate}

Die konkrete Umsetzung und genaue Spezifizierung der Tools wird in \cref{Tests} beschrieben. 

\subsection{Evaluationsszenarien} \label{Methodik.Szenarien}
Um die in \cref{Anforderungen} definierten Anforderungen zu validieren, werden zwei zu untersuchende 
Szenarien definiert:

\begin{enumerate}
    \crefalias{enumi}{scenario}
    \item \textbf{/S-01/ \phantomsection\label{req:S-01} Ermittlung der Performanzgrenzen:} In diesem Szenario wird jegliche künstliche Drosselung aufgehoben. Das Ziel ist es, die maximalen Durchsatzraten zu ermitteln. 
    Hierbei wird geprüft, wie effizient die Ressourcen unter Volllast genutzt werden, um die nicht-funktionalen Anforderungen \cref{req:NF-01,req:NF-05} zu untersuchen.
    \crefalias{enumi}{scenario}
    \item \textbf{/S-02/ \phantomsection\label{req:S-02} Simulation unter realen Parametern und \eng{Features}:} Um die Vergleichbarkeit zu praxisrelevanten Szenarien zu erhöhen, sollen Parameter gewählt werden, die für echte Internetscans 
    typisch sind. Dies soll die \eng{Performance}-Effizienz unter möglichst realen Bedingungen testen und die funktionale Anforderung \cref{req:F-08} untersuchen. 
    Dabei wird auch ein Augenmerk auf die Nutzung von \eng{Features} gelegt, die im Kontext
    eines realen Scans von Nutzen sind. Beispielsweise zur Verschleierung des Scans.
\end{enumerate}

\subsection{Metriken}
Der Begriff \enquote{\eng{Performance}-Effizienz} wird
gemäß der Norm \textit{ISO/IEC 25010} als die Fähigkeit eines Produkts, seine Funktionen innerhalb festgelegter 
Zeit- und Durchsatz-Parameter zu erfüllen und dabei die Ressourcen unter den gegebenen Bedingungen 
effizient zu nutzen, verstanden \cite{ISO/IEC_25010:2023}. 

Basierend auf der Definition werden folgende Metriken zur Quantifizierung herangezogen, wobei
die Paketrate den Durchsatzparameter und die CPU-Auslastung, sowie RAM-Verbrauch die Ressourcennutzung 
darstellen:

\begin{itemize}
    \item \textbf{/M-01/\phantomsection\label{req:M-01} Paketrate:} 
    Die Paketrate wird in Pakete pro Sekunde \eng{PPS} dargestellt und beschreibt die durchschnittliche Anzahl der erfolgreich an den 
    Netzwerkadapter übergebenen Pakete pro Sekunde. Da die Scanner nur sehr kleine Pakete verschicken ist die Paketrate in \eng{Performance}-orientierten Projekten dieser
    Art der limitierende Faktor \cite{TODO}, weshalb sie maßgeblich als Metrik für die \eng{Performance} dient.
    \item \textbf{/M-02/\phantomsection\label{req:M-02} CPU-Auslastung:} 
    Die CPU-Auslastung von hochperformanten Netzwerkanwendungen findet maßgeblich im \eng{User-Space}, im \eng{Kernel-Space} und im 
    \eng{SoftIRQ}\footnote{TODO erklären} 
    statt \cite{TODO}. deshalb sollten alle drei Bereiche betrachtet werden. Gemessen wird die prozentuale Auslastung der CPU-Kerne 
    in Bezug auf diese Metriken.
    \item \textbf{/M-03/\phantomsection\label{req:M-03} RAM-Verbauch:} 
    Der RAM-Verbrauch als zweiter primärer Teil der Ressourcenmetriken wird in Megabyte (MB) angegeben und stellt
den Anteil des physisch durch den Scanner belegten Arbeitsspeicher dar. 
\end{itemize}
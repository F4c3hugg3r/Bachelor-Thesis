% !TEX root = ../Thesis.tex
%%
%%  Hochschule für Technik und Wirtschaft Berlin --  Abschlussarbeit
%%
%% Kapitel 4 - Konzeption und Implementierung
%%
%%

\chapter{Methodik und Anforderungsanalyse} \label{Methodik}
Dieses Kapitel definiert die funktionalen und nicht-funktionalen Anforderungen an den zu entwickelnden Portscanner, 
beschreibt das gewählte Vorgehensmodell zur Umsetzung in Rust und legt das Untersuchungsdesign für die anschließende Evaluation fest.

\section{Anforderungsanalyse}

\subsection{Funktionale Anforderungen}
Die funktionalen Anforderungen definieren das Verhalten des Systems und die logischen Operationen, die der Scanner ausführen muss, 
um einen korrekten \texttt{SYN}-Scan durchzuführen.

\begin{itemize} 
    \item \textbf{/F-01/ Konstruktion valider \texttt{TCP-SYN}-Pakete:} Das System muss in der Lage sein, rohe TCP-Pakete zu konstruieren. 
    Dabei müssen \texttt{IP}-Header und \texttt{TCP}-Header manuell gesetzt und korrekte Prüfsummen berechnet werden, 
    um vom Zielsystem als legitime Verbindungsanfragen akzeptiert zu werden.

    \item \textbf{/F-02/ Senden von Paketen:} Das System muss in der Lage sein, TCP Pakete an andere Zielsysteme zu senden.
    Es sollten auch bei sehr hohem Durchsatz weniger als 10\% \textit{Packet-loss}\footnote{Verlorene Pakete}
    auftreten. Dies ist eine eher liberale Grenze und orientiert sich an den Verlustraten renommierter 
    hochleistungs-Scanner wie \textit{ZMap} (.TODO..\%) oder \textit{Masscan} (.TODO.. \%) inklusive eines kleinen Puffers aufgrund des begrenzten Umfanges einer
    Bachelorarbeit.
    
    \item \textbf{/F-03/ Empfang von Paketen:} Das System muss in der Lage sein, eingehende Netzwerkpakete
    abzufangen und zu untersuchen. Es sollten auch bei sehr hohem Durchsatz weniger als 5\% \textit{Packet-loss}
    auftreten. Dies ist eine eher liberale Grenze und orientiert sich an den Verlustraten renommierter 
    hochleistungs-Scanner wie \textit{ZMap} (.TODO..\%) oder \textit{Masscan} (.TODO.. \%) plus eines kleinen Puffers aufgrund des begrenzten Umfanges einer
    Bachelorarbeit.

    \item \textbf{/F-04/ \textit{Stateless-Scanning} (\texttt{SYN}-Cookie):} Um die Speicherung ausgehender Verbindungen zu eliminieren, muss der Scanner 
    verbindungsspezifische Informationen kryptografisch so in die \textit{Sequence Number} des ausgehenden Pakets kodieren, dass
    dieser zur Validierung für eingehende Pakete genutzt werden kann.

    \item \textbf{/F-05/ Validierung eingehender Antworten:} Die Empfangskomponente muss eingehende \texttt{SYN-ACK}-Pakete abfangen und validieren. 
    Dafür muss der Hash-Werte des \texttt{SYN}-Cookies korrekt erstellt und mit dem aus der \textit{Acknowledgement Number} extrahierten Wert verglichen werden.

    \item \textbf{/F-06/ Schließen der Verbindung auf Zielsystem:} Nach der Identifikation eines offenen Ports, sollte der Scanner 
    ein \texttt{RST}-Paket senden, um die halb-offene Verbindung auf dem Zielsystem sauber zu schließen und Ressourcenfreigabe zu ermöglichen.

    \item \textbf{/F-07/ Vermeidung von Duplikaten:} Die Endausgabe des Scan-Ergebnisses muss von Duplikaten bereinigt sein.
\end{itemize}

\subsection{Nicht-funktionale Anforderungen}
Die nicht-funktionalen Anforderungen stellen Qualitätsanforderungen dar und leiten sich primär aus dem Forschungsziel,
der Performance-Maximierung und der Verwendung von Rust ab. 

\begin{itemize} 
\item \textbf{/NF-01/ Maximierung des Durchsatzes:} Das System soll in der Lage sein, die verfügbare Bandbreite einer 
Standard-Gigabit-Schnittstelle bestmöglich auszunutzen. Zielgröße ist ein Sende-Durchsatz im mindestens fünfstelligen Paket-pro-Sekunde-Bereich, 
um mit etablierten Tools vergleichbar zu sein.

\item \textbf{/NF-02/ Asynchrone Architektur:} Die Anwendung muss Gebrauch von Lastverteilenden Maßnahmen in Form
von nebenläufiger Programmierung machen, um die Auslastung zu verteilen und Performance somit zu steigern.

\item \textbf{/NF-03/ Nutzung moderner Kernel-Mechanismen:} 
    Zur Steigerung der Performance und Forschungsrelevanz sollen fortschrittliche Linux-Mechanismen zur Paketverarbeitung, 
    spezifisch \texttt{AF\_XDP} oder \texttt{eBPF}, evaluiert und implementiert werden.

\item \textbf{/NF-04/ Speichersicherheit:} 
Die Gesamtarchitektur soll die Sicherheitsgarantien von Rust wahren. \texttt{unsafe}-Blöcke können, falls nötig, genutzt werden 
(z.B. für systemnahe Netzwerkoperationen) sollten aber möglichst vermieden oder durch die Nutzung von externen Bibliotheken  
ersetzt werden, da diese intern \texttt{unsafe}-Blöcke häufig sicher kapseln \cite{RustBelt}.

\item \textbf{/NF-05/ Minimale Ressourcennutzung:} Die Architektur und gewählten Technologien sollten die Ressourcennutzung
(CPU-Zeit, RAM Verbrauch) neben der Durchsatzgeschwindigkeit nach /NF-01/ priorisieren und somit minimieren.
Keine der beiden Anforderungen darf aufgrund der anderen stark vernachlässigt werden. 
\end{itemize}

\section{Vorgehensmodell der Entwicklung}
Für die Realisierung wird ein evaluationsgetriebener, prototypischer Ansatz gewählt. Aufgrund der Komplexität asynchroner 
Netzwerkprogrammierung, sowie der Vielzahl möglicher Technologien, empfiehlt es sich, verschiedene Wege auszuprobieren. Dies
hilft bei der Schaffung einer realistischen Vergleichsbasis zwischen den verschiedenen Technologien und bietet die Möglichkeit,
Ausweichlösungen bei Problemen wie z.B. Performance-Engpässe zu finden oder die tatsächliche Notwendigkeit komplexer 
Technologien zu validieren. Die Entwicklung teilt sich in 2 Hauptphasen:

\begin{enumerate}
    \item \textbf{Basisimplementierung:} In der Basisimplementierung wird ein vollumfänglicher \texttt{SYN}-Scanner implementiert.
    Dies soll zum einen als \textit{Proof of Concept}, und zum anderen als erste Vergleichsgrundlage dienen. Für 
    das Senden, werden bereits die Möglichkeit des Sendens mit \texttt{AF\_PACKET} oder \texttt{AF\_XDP} und jeweils auch in \textit{Batches}\footnote{Gruppen von Paketen}
    oder Einzelpaketen implementiert. Für das Empfangen wird der \textit{Crate}\footnote{Bezeichnung für Bibliothek im Rust-Ökosystem} \texttt{pcap} genutzt,
    welcher das Empfangen von Paketen abstrahiert, allerdings nicht den Netzwerkstack des Linux-Kernels umgeht. 

    Außerdem wird die für die nötige Struktur um die Komponenten zu vernetzen, sowie die Pakete korrekt zu erstellen
    und verarbeiten implementiert, sodass das Programm am Ende einen funktionierenden \texttt{SYN}-Scan vollführen kann.
    Zusätzlich wird auch hier schon auf die möglichst performante Umsetzung der gesamten Struktur, spezifischer
    Umsetzungen und der Wahl von Crates Wert gelegt. 

    \item \textbf{Optimierung des Empfangspfades durch Nutzung von \texttt{eBPF}:} Basierend auf den Messergebnissen der ersten Phase 
    wird die Empfangskomponente hier grundsätzlich verändert, um eine hochperformante, sowie effizienten Paketempfang und
    Paketauswertung zu gewährleisten. Dafür wird ein \texttt{eBPF} in Verbindung mit einem \texttt{RingBuf} zur Protokollierung in Verbindung
    mit einem \texttt{XDP} Programm statt des \texttt{pcap}-Crates genutzt. maybe TODO "bla bla pcap kann ab gewisser Geschwindigkeit nicht mithalten
    oder pcap ist zu ineffizient und/oder rst werden immer automatisch gesendet". Außerdem werden weitere, kleinere
    Performanz-steigernde Maßnahmen addressiert.
\end{enumerate}


\section{Untersuchungsdesign}
Hier müssen Sie definieren, wie Sie vergleichen.
-	Definition der Metriken: Warum messen Sie pps? Warum RAM? Wie messen Sie CPU (User vs. System Time)?
-	Definition der Testumgebung: (Hier nur theoretisch – z.B. "Isolierte Laborumgebung um Netzwerrauschen zu minimieren").




% !TEX root = ../Thesis.tex
%%
%%  Hochschule für Technik und Wirtschaft Berlin --  Abschlussarbeit
%%
%% Kapitel 5 - Evaluation
%%
%%

\chapter{Testumgebung und Durchführung} \label{Tests}
% TODO hier Text einfügen

\section{Versuchsaufbau} 
\subsection{Hardware-Spezifikation}
Für die Umsetzung des Testes wurden die in \ref{tab:specs_combined} spezifizierten Systeme verwendet. Es wurden keine Veränderungen 
vorgenommen, die das Betriebssystem von seinem nativen Zustand abbringen, um Anforderung \hyperref[req:F-10]{/F-10/} zu genügen. 
Die einzige Konfigurationsmaßnahme außerhalb des Programmes ist eine softwareseitige Anpassung der Sende-, sowie Empfangsringe der Netzwerkkarte auf 
das Maximum der bereitgestellten Hardware. Dies ist in diesem Szenario zu empfehlen, da es sich explizit um ein Hochleistungsszenario
handelt. Das Anpassen der genutzten Puffer dient dem Abfangen von Lastspitzen, indem mehr Pakete vom Netzwerkkartentreiber gepuffert werden können. 
Für die genutzten Netzwerkkarten bedeutet dies konkret eine Erhöhung der sogenannten Deskriptoren von 256 auf 4096 (siehe \textbf{Anhang \ref{app:ethtool_output}}).

\begin{table}[htbp]
\centering
\begin{tabularx}{\textwidth}{|l|X|X|} \hline
\textbf{Komponente} & \textbf{Scanner-Knoten} & \textbf{Ziel-Knoten} \\ \hline
\multicolumn{3}{|c|}{\textbf{\textit{Hardware}}} \\ \hline
CPU & Intel Core i5-11400F \newline (6 Kerne, 2.6 GHz) & Intel Core i5-7500 \newline (4 Kerne, 3.4 GHz) \\ \hline
RAM & 48 GB DDR4 (2667 MHz) & 8 GB DDR4 (2400 MHz)\\ \hline
Netzwerkkarte & Intel I210-T1 (Gbit) & Intel I350-T2 (Gbit) \\ \hline
Verbindung & \multicolumn{2}{c|}{Direktverbindung via CAT 6 S/FTP Kabel (Gbit)} \\ \hline
\multicolumn{3}{|c|}{\textbf{\textit{Software}}} \\ \hline
Betriebssystem & Ubuntu 24.04.3 LTS & Ubuntu 24.04.3 LTS \\ \hline
Kernel & 6.14.0-37-generic & 6.14.0-37-generic \\ \hline
Treiber & igb (6.14.0-37) & igb (6.14.0-37) \\ \hline
\end{tabularx}
\caption{Hardware- und Software-Spezifikationen der Testumgebung im Vergleich}
\label{tab:specs_combined}
\end{table}

Um die Performanz des Scanners isoliert von externen Störfaktoren zu evaluieren, wird ein dedizierter 
Laboraufbau gewählt. Dafür werden zwei Geräte per Ethernet-Kabel direkt miteinander verbunden:

\begin{enumerate}
    \item \textbf{Der Scanner-Knoten:} Das System, auf dem der Prototyp ausgeführt wird und welches die Systemmetriken erfasst.
    \item \textbf{Der Ziel-Knoten:} Ein System, welches ein Zielnetzwerk simuliert. 
\end{enumerate}

Für die Schaffung einer einheitlichen und reproduzierbare Messgrundlage, welche menschliche Fehler möglichst vermeidet, wurden für die Szenarien 
\ref{req:S-01} und \ref{req:S-02} (siehe \ref{Methodik}) Python-Programme erstellt, mit
welchen der Ablauf, die Datenerfassung und Datenaufbereitung automatisiert wird. Dieser Ansatz gewährleistet konsistente
Rahmenbedingungen in Form von einheitlichen Pausenzeiten und der einheitlichen Erfassung der Systemressourcen.

Die Messung erfolgt mit einer Abtastrate von $10\,\text{Hz}$ (Intervall $t=0,1\,\text{s}$). 
Da die Scanner Subprozesse und mehrere \eng{Threads} starten, wird die gesamte Systemlast gemessen.
Das \eng{Benchmarking}-Programm verzichtet auf \eng{High-Level}-Tools, um den \eng{Overhead} der Messung selbst 
zu minimieren und liest die erforderlichen Kernel-Statistiken direkt aus \texttt{procfs}.
Dies ist ein virtuelles Dateisystem, welches der Anzeige und Änderung von Systemparametern dient \cite{kernel.org/doc/Documentation/filesystems/proc.txt}
und eine direkte Schnittstelle zu den internen Datenstrukturen des Linux-Kernels bietet. 

Um mögliche Fluktuationen in der Grundlast zu vermeiden, wird jeglicher Zugang
zum Internet geschlossen und kein weiterer Prozess abseits des Benchmarking-Programmes manuell gestartet.
Jedes Szenario wird in fünf unabhängigen Iterationen durchgeführt, sodass statistische Ausreißer weniger ins Gewicht fallen. 

\subsection{Aufbau des Ziel-Knotens}
Der Zielknoten besteht ähnlich wie die Empfangslogik des Scanners aus einem mithilfe des \texttt{aya} \eng{Crates} erstellten \texttt{eBPF}-Programmes, welches die eingehenden Pakete mittels
Zeiger-Operationen parst, anschließend validiert und bei Erfolg ein Antwortpaket via \texttt{XDP\_TX} versendet. Allerdings wird hier geprüft, ob es sich um ein \texttt{IPv4-SYN}-Paket
handelt und anschließend eine darauf zugeschnittene \texttt{SYN-ACK}-Antwort statt eines \texttt{RST}-Paketes versendet. Die Modifikation passiert auch hier am selben Paket ohne dieses
zuvor zu kopieren. 

Zur Steuerung der Antwortwahrscheinlichkeit ist ein beim Start des Programmes veränderbarer Parameter integriert, welcher über die Kommandozeile übergeben wird. 
Wenn ein valides Paket erhalten wurde,
wird eine Zufallszahl generiert und in Verbindung mit der eingegebenen Prozentzahl genutzt, um zu entscheiden, ob eine Antwort gesendet wird oder nicht.

Auch die Statistiken werden mit der gleichen Datenstruktur wie beim Scanner - dem \texttt{PerCpuArray} - ermittelt, um eine lockfreie, effiziente Erhebung zu gewährleisten.
Es werden folgende Statistiken ermittelt, um die Funktionsweise des Scanners und des Laboraufbaus zu validieren:

\begin{enumerate}
    \itemsep 0pt
    \item \textbf{Empfangene Pakete gesamt} 
    \item \textbf{Valide \texttt{SYN}-Pakete}
    \item \textbf{Gesendete Antworten}
\end{enumerate}

\section{Versuchsablauf}
Die \eng{Benchmark Suite} \ref{TODO anhang} startet die Scanner-Prozesse nacheinander und misst mit einem parallelen \eng{Monitoring}-\eng{Thread} die
Systemressourcen in folgenden Phasen:

\begin{enumerate}
    \item \textbf{Ruhezustands-Messung:} Vor den eigentlichen Tests wird über einen Zeitraum von fünf Sekunden der Systemzustand ohne Last gemessen. 
    Dieser Durchschnittswert dient als Referenzpunkt, um das Grundrauschen des Betriebssystems später herausrechnen zu können.
    \item \textbf{Aufzeichnung:} Die Aufzeichnung der Metriken beginnt eine Sekunde vor dem Prozessstart, 
    um das Anlaufverhalten und Initialisierungsspitzen der Scanner vollständig zu erfassen.
    \item \textbf{Aktive Phase:} Während der Scanner läuft, überwacht ein \eng{Watchdog}-Algorithmus\footnote{TODO} den ausgehenden Datenverkehr. 
    Der Scanner gilt als aktiv, sobald die Senderate 100 \eng{PPS} überschreitet und als inaktiv, sobald die Grenze wieder unterschritten wird.
    \item \textbf{Externes Beenden:} Sollte die Rate nach dem Start für mehr als sieben Sekunden unter einen Schwellenwert von 100 \eng{PPS} fallen, 
    wird der Prozess terminiert. Dies ist notwendig, da Masscan sich häufig nicht von alleine beendet.
\end{enumerate}

\subsection{Datenaufbereitung und Bereinigung} \label{Versuchsablauf.Datenaufbereitung}
Um die Scanergebnisse zu plausibilisieren wird zusätzlich via Python-Programm \ref{TODO} die Anzahl der ausgegebenen Ergebnisse gezählt. 
Zur Validierung werden im ersten Szenario die Werte zusätzlich mit den empfangenen Paketen des Ziel-Knotens und den theoretischen Werten verglichen.
Da keine automatische Synchronisierung der Knoten stattfindet, wird dieses Szenario also manuell durchgeführt.

Die Rohdaten werden nach der Messung mithilfe des Programmes \ref{TODO} statistisch bereinigt und visualisiert. 
Eine einfache Mittelwertbildung über die gesamte Laufzeit alleine ist nicht zielführend, da Start- und Stopp-Phasen die Ergebnisse verzerren würden. 
Stattdessen werden die Ergebnisse mit folgenden Vorgehensweisen aufbereitet:

\begin{itemize}
    \item \textbf{Isolation der Hochlastphase:} Für die Berechnung des durchschnittlichen Durchsatzes und der Effizienz (\eng{PPS}/CPU Auslastung) 
    werden nur jene Zeitfenster berücksichtigt, in denen der Scanner aktiv sendet.
    \item \textbf{Netto-Ressourcenberechnung:} Von den gemessenen CPU- und RAM-Werten wird der in Phase (siehe \ref{Methodik}) ermittelte \eng{Baseline}-Wert subtrahiert.
    Dies stellt sicher, dass die dargestellten Ergebnisse ausschließlich den Ressourcenbedarf des Scanners abbilden und unabhängig von Hintergrundprozessen des Betriebssystems sind.
\end{itemize}

Aus den bereinigten Daten werden anschließend Diagramme und Tabellen via Python \texttt{matplotlib} generiert, welche die Daten in anschaulicher Weise darstellen.

\section{Inkompatibilitäten und Limitierungen}
% TODO hier text schreiben

\subsection{\eng{Zero-Copy}-Modus}
Da finanziell bedingt nur die Möglichkeit besteht die in \ref{tab:specs_combined} angegebene Hardware zu nutzen, muss für die Evaluation eine 
Einschränkung gemacht werden. Der äußerst effiziente \eng{Zero-Copy}-Modus ist bei der Nutzung von Netzwerkkarten mit dem \texttt{igb}-Treiber eingeschränkt. Aufgrund
der niedrigen Anzahl an Sende- und Empfangsringen auf der Netzwerkkarte können dem \texttt{XDP}-Programm keine dedizierten \eng{Queues} \footnote{Warteschlangen} zugeteilt werden. Dies hat zur Folge,
dass sich die Sende-Ringe mit dem Betriebssystem geteilt werden müssen und es somit zu sogenannter \eng{Lock Contention} \footnote{Der Zugriff auf den Ring muss bei jeder Operation 
ausgehandelt werden, sollten mehrere Parteien ihn gleichzeitig nutzen wollen} kommen kann \cite{igb}. Dies ist sehr Ressourcenaufwändig und führte in den Tests dazu,
dass der \texttt{AF\_XDP}-\eng{Socket} und das \texttt{eBPF}-Programm um den Zugriff auf einen Sendering konkurrierten, sobald ein \texttt{SYN-ACK}-Paket
über die gleiche \eng{Queue} einging, über die auch versendet wurde. Dies ist unvermeidbar, da ein \texttt{RST}-Paket per \texttt{XDP\_TX} zwangsweise über die selbe \eng{Queue} verschickt wird, 
über die es eingeht \cite{TODO} und die Pakete durch den ...TODO... gleichmäßig auf alle \eng{Queues} verteilt werden \cite{TODO}.
Aufgrund der sehr hohen Senderate und der vergleichbar kleinen Puffer der Netzwerkkartenringe \footnote{Für die Intel I210 und I350 sind es maximal 4096 Deskriptoren (siehe \ref{app:ethtool_output})}
laufen dieser durch die \eng{lock contention} voll und blockieren das Versenden der \texttt{RST}-Antworten in dieser \eng{Queue}.

Diese Vermutungen wurden dadurch gestützt, dass bei den Tests unabhängig der gesendeten Menge immer 25\% der \texttt{RST}-Pakete beim Ziel-Knoten fehlten. 
Da das Scanner-System 4 \eng{Queues} besitzt und eine \eng{Queue} für das Senden genutzt wurde entstehen somit ein Viertel-, beziehungsweise 25\% der Pakete, welche aufgrund der \eng{Lock Contention}
verloren gingen. Tests mit Beschränkung der \eng{Queue}-Anzahl mithilfe des folgenden Befehls: \texttt{sudo ethtool -L enp6s0 combined x} auf die Anzahl x bestätigten dies auch. 
Bei einer \eng{Queue} wurden rund 100\% und bei 3 \eng{Queues} rund 33\% der \texttt{RST}-Pakete nicht gesendet. 

Aufgrund dessen wird der \eng{Zero-Copy}-Modus ausschließlich ohne das Senden von \texttt{RST}-Paketen getestet.

\subsection{Paketrate}
Die Durchsatzrate ist durch die genutzten Netzwerkkarten sowie dem LAN-Kabel auf das Limit einer Gigabit-Verbindung beschränkt. Die Tests zur maximalen Durchsatzrate 
können deshalb nur eingeschränkt durchgeführt werden. Das theoretische Limit einer Gigabit-Leitung liegt nach IEEE 802.3 \cite{IEEE_Standard_for_Ethernet}, 
bei einer Paketgröße von 64 Byte plus 20 Byte \eng{Overhead}, bei 1,488 Millionen \eng{PPS}. Deshalb wird der Fokus dort verstärkt auf die gemessene Ressourcenauslastung gelegt.

\section{Umsetzung der Testszenarien} \label{Umsetzung_Testszenarien}
Im Folgenden wird die technische Umsetzung der in Abschnitt \ref{Methodik} definierten Szenarien beschrieben. 
Die Szenarien \ref{req:S-02} und \ref{req:S-03} (siehe \ref{Methodik}) werden mit allen drei zu evaluierenden Scanner-Varianten (Rust-XDP-Copy, Rust-XDP-ZeroCopy, Rust-\texttt{AF\_PACKET}) sowie den Vergleichstools 
(ZMap, Masscan) durchgeführt. Szenario \ref{req:S-01} (siehe \ref{Methodik}) wird nur mit den Rust-Scannern durchgeführt.

In allen Szenarien wurden 64 verschiedene Source-IP-Adressen genutzt, da dies ein essenzieller Faktor zur Erhöhung der Antwortwahrscheinlichkeit 
in realen Hochgeschwindigkeits-Scan-Szenarien darstellt \cite{Durumeric_Adrian_Stephens_Wustrow_Halderman_2024}. Außerdem antwortet der Ziel-Knoten
auf 20\% der Pakete mit validen SYN-ACK Antworten. Dies ist im Vergleich zur realistischen Antwortrate wenn man den kompletten IPv4-Raum scannt ein 
sehr hoch angesetzter Wert \cite{Durumeric_Adrian_Stephens_Wustrow_Halderman_2024}, soll aber die Funktionsfähigkeit für Spezialfälle sicherstellen. 

\subsubsection{Szenario \ref{req:S-01} - Anforderungsvalidierung}
Der erste Test dient als \eng{Proof of Concept}. Hierbei wird ein moderater IP-Adressraum (\texttt{/20}) gescannt, wobei die Senderate künstlich auf $100.000$ \eng{PPS} limitiert wird. 
Der Ziel-Knoten antwortet auf alle IP-Adressen welche mit einer geraden Zahl enden, um die anschließende Auswertung zu vereinfachen.
Um die funktionalen Anforderungen zu validieren, das Senden von \texttt{RST}-Paketen und die IP-Deduplizierung aktiviert.

\subsubsection{Szenario \ref{req:S-02} - Performanzgrenze}
Um die Performance und Effizienz der Scanner zu validieren, werden alle bremsenden Faktoren, die nicht essenziell für das reine Versenden und Empfangen 
von Paketen sind, deaktiviert.
\begin{itemize}
    \itemsep 0pt
    \item \textbf{Senderate:} Die Senderate wird so gewählt, dass sie das theoretische Limit der Gigabit-Leitung übersteigt.
    \item \textbf{Features:} Es wird auf rechenintensive Funktionen wie die Deduplizierung von Antworten, sowie das Senden von \texttt{RST}-Antworten verzichtet.
    \item \textbf{IP-Raum:} Als Ziel dient ein \texttt{/6}-Netzwerk\footnote{67.108.864 Millionen IP-Adressen}, um eine hinreichend lange Laufzeit für die Erfassung stabiler Messwerte zu gewährleisten.
%vergessen    \item \textbf{Source-Ports:} Ein einzelner Quellport ($60000$) wird genutzt, um die Vergleichbarkeit mit ZMap und Masscan zu gewährleisten.
%    Das Senden von \texttt{RST}-Antworten geschieht bei der Nutzung der beiden Scanner durch den \texttt{Kernel} und kann nur dadurch verhindert werden, dass via \texttt{iptables}-Regel\footnote{\texttt{sudo iptables -A INPUT -p tcp --dport 60000 -j DROP}} 
%    eingehende Pakete auf diesem Port verworfen werden. 
\end{itemize}

\subsubsection{Szenario \ref{req:S-03} - Reales Szenario}
Das dritte Szenario simuliert einen praxisnahen Scan-Vorgang. Hierbei werden Parameter gewählt, die helfen, Sicherheitsmechanismen zu umgehen oder 
die Zuordnung von Antworten zu erleichtern.
\begin{itemize}
    \itemsep 0pt
    \item \textbf{Senderate:} Die Senderate wird auf $500.000$ \eng{PPS} fixiert, da übermäßig hohe Raten die Wahrscheinlichkeit erhöhen, dass Scan-Muster von \eng{Firewalls} oder \eng{IPS} erkannt und blockiert werden.
    \item \textbf{Ports:} Der Scan erfolgt parallel auf den Ports $80$ und $443$, um das Scan-Verhalten zu diversifizieren und weiter zu verschleiern.
    \item \textbf{Quellports:} Es wird ein Bereich von 128 \eng{Source-Ports} ($60000-60127$) verwendet. Dies dient der besseren Lastverteilung auf der Empfängerseite und Verschleierung des Scans.
    \item \textbf{IP-Raum:} Der Zielbereich wird auf ein \texttt{/10}-Netzwerk\footnote{4.194.304 Millionen IP-Adressen} beschränkt, um die Gesamtdauer des Tests in einem praktikablen Rahmen zu halten.
\end{itemize}
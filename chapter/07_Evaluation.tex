% !TEX root = ../Thesis.tex
%%
%%  Hochschule für Technik und Wirtschaft Berlin --  Abschlussarbeit
%%
%% Kapitel 7 Fazit und Ausblick
%%
%%

\chapter{Evaluation und Ausblick} \label{Evaluation}

In diesem Kapitel werden die in der Testumgebung ermittelten Messergebnisse vorgestellt, analysiert und diskutiert. Ziel ist es, die Leistungsfähigkeit des implementierten Rust-Scanners im Vergleich zu etablierten Tools zu bewerten und die Erfüllung der definierten Anforderungen zu überprüfen. Abschließend wird ein Ausblick auf mögliche Weiterentwicklungen gegeben.

\section{Darstellung der Messergebnisse}
% Hier präsentieren Sie die reinen Daten (Diagramme, Tabellen).
% Orientieren Sie sich an den Szenarien aus Kap. 6 (S-01, S-02, S-03).

\subsection{Ergebnisse S-01: Anforderungsvalidierung}
% War der Scanner stabil? Wurden alle Pakete korrekt gebaut?
% Gab es Abstürze oder Memory Leaks bei moderater Last?

\subsection{Ergebnisse S-02: Performanzgrenzen}
% Das wichtigste Kapitel. Vergleich der Durchsatzraten (PPS).
% Vergleich Rust-XDP (Copy vs ZeroCopy) vs Rust-AF_PACKET vs ZMap vs Masscan.
% Grafiken: Durchsatz über Zeit, Drop-Raten.

\subsection{Ergebnisse S-03: Reales Szenario}
% Wie verhält sich der Scanner bei Antwortraten von 20%?
% Funktioniert die Duplikaterkennung?

\section{Diskussion der Ergebnisse}
% Hier interpretieren Sie die Zahlen aus 7.1.

\subsection{Analyse des Durchsatzes und der Latenz}
% Warum ist XDP schneller als AF_PACKET?
% Wo liegen die Flaschenhälse (CPU, PCIe-Bus, Kernel)?

\subsection{Ressourceneffizienz (CPU und RAM)}
% Bezug auf Anforderung /M-02/ und /M-03/.
% Vergleich Rust vs. C (ZMap/Masscan).
% Diskussion des Overheads durch Tokio/Async vs. Threading.

\subsection{Vergleich mit dem Stand der Technik}
% Expliziter Vergleich mit ZMap und Masscan.
% Wo steht Ihre Lösung? (Schneller? Sicherer? Ressourcenhungriger?)

\section{Abgleich mit den Anforderungen}
% Gehen Sie die Liste aus Kap. 4 durch (/F-01/ bis /F-09/ und /NF-xx/).
% Eine Tabelle eignet sich hier gut (Anforderung | Status | Bemerkung).

\section{Ausblick}
% Was konnte in dieser Arbeit nicht umgesetzt werden?
% Ideen: IPv6 Support, Distributed Scanning, GUI, Output-Module (JSON/Redis).

\section{Fazit}
% Zusammenfassung der gesamten Arbeit.
% Forschungsfrage beantworten: Ist Rust/eBPF geeignet für High-Speed Scanning?
% !TEX root = ../Thesis.tex
%%
%%  Hochschule für Technik und Wirtschaft Berlin --  Abschlussarbeit
%%
%% Kapitel 7 Fazit und Ausblick
%%
%%

\chapter{Fazit und Ausblick} \label{Fazit}

In diesem Kapitel wird die erstellte Lösung begutachtet. Es werden Anwendungsmöglichkeiten, allgemein von Smart Objects und speziell für die Implementierung, behandelt. Mögliche Erweiterungen werden vorgeschlagen und ein Fazit der Arbeit wird gezogen.

\begin{table}[btp]
\small
\caption{Materialkosten Steckdose}
\begin{tabular}{|r|l|l|p{2cm}|p{2cm}|}\hline
  \textbf{\#} & \textbf{Bauteil} & \textbf{Artikelnummer} & \textbf{Einzel-preis in~€} & \textbf{Massen-preis in~€} \\ \hline
  \multicolumn{5}{|l|}{Zigbit-Platine} \\ \hline
  1 & ATZB-24-A2 & 556-ATZB-24-A2 & 25,43 & 14,42 \\ \hline
  1 & Widerstand 100kOhm & 71-CMF60100K00FKEB & 0,206 & 0,099 \\ \hline
  1 & Widerstand 1kOhm & 71-CMF551K0000FHEK  & 0,116 & 0,005 \\ \hline
  1 & Kondensator 3,3uF & 667-EEU-HD1H3R3 & 0,165 & 0,104 \\ \hline
  1 & Spannungsregler LP2950CZ-3.0 & 926-2950CZ-3.0/NOPB & 0,676 & 0,27 \\ \hline
  1 & Steckverbinder FFC 6 Pin & 538-52271-0679 & 1,30 & 0,583 \\ \hline
  2 & Steckverbinder FFC 18 Pin & 538-52271-1879 & 1,71 & 0,94 \\ \hline
  \multicolumn{5}{|l|}{Steckdosen-Platine} \\ \hline
  1 & Kondensator X2 275V & 80-R46KN368050M2M & 0,66 & 0,263 \\ \hline
  1 & Widerstand 560Ohm, 5Watt & 594-AC05W560R0J & 0,38 & 0,198 \\ \hline
  2 & Widerstand 1MOhm & 594-MRS251M1\%TR & 0,074 & 0,038 \\ \hline
  2 & Gleichrichterdiode & 512-1N4004 & 0,076 & 0,025 \\ \hline
  2 & Zener-Diode 24V & 512-1N4749ATR & 0,203 & 0,036 \\ \hline
  1 & Kondensator 470uF & 667-EEU-FR1E471YB & 0,248 & 0,152 \\ \hline
  1 & NPN-Transistor BC547B & 512-BC547B & 0,186 & 0,05 \\ \hline
  1 & Widerstand 220kOhm & 271-220K-RC & 0,125 & 0,012 \\ \hline
  \multicolumn{5}{|l|}{Steckdosen-Gehäuse} \\ \hline
  1 & Tchibo Digitale Zeitschaltuhr & 4 043002 669758 & 5,99 & 5,99 \\ \hline
  \multicolumn{5}{|l|}{} \\ \hline
  \multicolumn{3}{|l|}{\textbf{Gesamtpreis}} & \textbf{38,61} & \textbf{24,27} \\ \hline
\end{tabular}
\label{tab:hardwarekosten}
\end{table}

Die Gesamtmaterialkosten der verwendeten Bauteile betragen 38,61€. Nicht berücksichtigt sind Verbrauchsmaterialien. Dazu zählen die Leiterplatte für die Zigbit-Platine, die aus einer vorhandenen Lochrasterplatine ausgesägt wurde, Leitungen für die Verkabelung und Lötzinn. Diese Kosten sollen hier vernachlässigt werden. Bei einer Kalkulation für eine Fertigung in einem Unternehmen müssen diese als Materialgemeinkostenzuschlag zu den Materialkosten hinzugefügt werden. Zusammen mit den Fertigungskosten, also die Arbeitskosten, die zum Fertigen der Steckdose notwendig sind, und den Verwaltungs- und Vertriebsgemeinkostenzuschlag ergeben sich die Selbstkosten je Stück.



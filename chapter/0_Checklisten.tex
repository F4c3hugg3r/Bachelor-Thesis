% Checklist Formatierung (wichtig: einheitlich!):

% 1. Eigennamen (Sprachen, Firmen): 
% "Rust", "C", "Linux", "Windows" sind Eigennamen. Auch Projekte oder Tools wie "ZMap" gelten zu Eigennamen, 
% wenn sie nicht im Kontext der Programmierung besprochen werden.
% -> Normale Schrift (kein \texttt, kein \textit).

% 2. \textit{} (Kursiv) wird verwendet für:
% -> Ersteinführung von DEUTSCHEN Fachbegriffen: "Dies nennt man \textit{Lebenszeit}."
% -> Mathematische Variablen im Fließtext: "Der Wert $x$ wird um $y$ erhöht."
% -> NICHT für Einheiten (falsch: $100 ms$, richtig: $100\,\text{ms}$ oder siunitx).

% 3. \eng{} (oder \textit{}) für fremdsprachige Begriffe (Vorgabe Erstprüfer):
% -> Für Begriffe, die nicht im Duden stehen (Thread, Crate, Scheduling, Overhead).
% -> WICHTIG: Grammatikalisch als DEUTSCHE Substantive behandeln (Großschreibung!).
% Richtig: „Der \eng{Thread} bricht ab.“ (Groß, da Substantiv).
% Richtig: „Das \eng{Polling} verursacht Last.“
% Falsch: „Der \eng{thread} bricht ab.“

% 4. \texttt{} (Monospace) für alles Computerlesbare:
% -> Befehle & Tools: „Der Compiler \texttt{rustc}...“, „Das Tool \texttt{ethtool}...“
% -> Dateinamen & Pfade: „In \texttt{/proc/net/dev} steht...“, „Die Datei \texttt{main.rs}...“
% -> Code-Schnipsel & Funktionen: „Die Funktion \texttt{unwrap()}...“, „\texttt{println!}...“
% -> Keywords & Flags: „Das Flag \texttt{--release}...“, „Keyword \texttt{impl}...“
% -> Konkrete Artefakte (Pakete/Crates): „Das Crate \texttt{serde} wird importiert.“
%    (Merke: Crate-Namen meist kleingeschrieben, Konzept "Crate" groß).

% 5. URLs (WICHTIG für Fußnoten/Quellen):
% -> Nutze \url{https://...} (benötigt \usepackage{url} oder hyperref).
% -> Nicht einfach \texttt{} nehmen, da \url{} den Zeilenumbruch besser kann.
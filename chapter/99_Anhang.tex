% !TEX root = ../Thesis.tex
%%
%%  Hochschule für Technik und Wirtschaft Berlin --  Abschlussarbeit
%%
%% Anhang
%%
%%%%%%%%%%%%%%%%%%%%%%%%%%%%%%%%%%%%%%%%%%%%%%%%%%%%%%%%%%%%%%%%%%%%%


\chapter{Ergänzende Systeminformationen}

\section{Netzwerkkarten-Konfiguration (Ethtool)}
\label{app:ethtool_output}

Der folgende Auszug zeigt die Standard-Konfiguration der Netzwerkschnittstelle \texttt{enp6s0} vor der Optimierung des Ring-Buffers.

\begin{verbatim}
Ring parameters for enp6s0:
Pre-set maximums:
RX:             4096
RX Mini:        n/a
RX Jumbo:       n/a
TX:             4096
TX push buff len:       n/a
Current hardware settings:
RX:             256
RX Mini:        n/a
RX Jumbo:       n/a
TX:             256
RX Buf Len:     n/a
CQE Size:       n/a
TX Push:        off
RX Push:        off
TX push buff len:       n/a
TCP data split:         n/a
\end{verbatim}

\section{Scanergebnisse \cref{req:S-01}}
\label{app:scan_results}

\begin{figure}[htbp]
	\centering
	\includegraphics[width=\textwidth]{pictures/vergleich_balken_4_effizienz.png}
	\caption{Effizienz der SYN-Scanner im Benchmark (aktiv)}
	\label{fig:performance_efficiency_diag}
\end{figure}

\begin{figure}[htbp]
	\centering
	\includegraphics[width=\textwidth]{pictures/vergleich_balken_2b_cpu_total_gesamt.png}
	\caption{CPU-Auslastung der SYN-Scanner im Benchmark (gesamt)}
	\label{fig:cpu_efficiency_diag}
\end{figure}

\begin{figure}[htbp]
	\centering
	\includegraphics[width=\textwidth]{pictures/vergleich_zeitreihe_3_ram.png}
	\caption{Effizienz der SYN-Scanner im Benchmark (gesamt)}
	\label{fig:ram_efficiency_diag}
\end{figure}